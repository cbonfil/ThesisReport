% Chapter Template

\chapter{Conclusion} % Main chapter title

\label{Chapter7} % Change X to a consecutive number; for referencing this chapter elsewhere, use \ref{ChapterX}

\lhead{Chapter 7. \emph{Conclusion}} % Change X to a consecutive number; this is for the header on each page - perhaps a shortened title

\begin{doublespace}

By looking at the experimental results versus the simulation it is possible to conclude that the memristor model captures a variety of physical effects that are present in the memristor and produces satisfactory results. The change in the conductivity is captured through the exchange of lithium ions between the electrolyte and the PEDOT:PSS and the undoping of PEDOT:PSS. The transient simulation results match the experimental results behaviorally for both single strip and notched memristor. Although the simulation matches the experimental results, this model requires more work and can be improved through better numerical schemes, physical models and thorough experimentation. There are also a few limitations of this model due to its size and physical features.

Both 1-D and 2-D simulations are effective in capturing the essential physical effects such as the movement of ions and the change in conductivity over time. 1-D simulations have an advantage over 2-D simulations when it comes to computational speed. 1-D simulations take between 10 to 30 min on average (on a quad core Intel i7 CPU) where 2-D simulations take a minimum of 2 hours to complete. Even though 1-D simulations run faster than 2-D simulations, they also require fine tuning of the lithium mobility in order to get a comparable accuracy. Since the electrolyte is never fully simulated in 1-D, it cannot match the accuracy of 2-D simulations.

Even though the model that is used for the simulation of the memristor is very basic, the computational load to generate a transient response is very high in both 1-D and 2-D and the simulation in 3-D is not feasible using a regular desktop computer. Modeling of additional physical effects can slow down the simulation even further. However, the model is still suitable for any structures that can be simulated in 2-D.  

Another drawback of this model is the inability to simulate using high charge densities due to the debye length. Mesh density can only be increased up to a certain value before the memory requirements of the simulation becomes unmanageable for a regular computer. Fortunately using a low charge density does not render the model useless since simulations in section 5.4.3 showed that it does not drastically alter the physics of the problem. Simulations using low charge densities can be scaled up using appropriate fitting parameters.  
 
The model that is developed in this thesis is open for improvements on carrier transport models. A constant bulk mobility is used for holes in this model which is not an entirely accurate representation of the actual transport behavior of the holes which move from site to site via variable range hopping. Addition of the lithium ions into the PEDOT:PSS not only reduces the number of available holes but also decreases the amount of available sites through holes can move. A variable range hopping mechanism and some thermal effects can be added into this model for a more complete simulation. Also PEDOT:PSS is a disordered material and the way it was deposited on a substrate can make a big difference on the hole transport. Anisotropic hole mobilty can be implemented in order to account for this issue.

The results showed that it is possible to simulate a memristor using a simple finite difference method. This model can be a very useful tool in understanding the way the ions and the holes move in PEDOT:PSS. The model developed in this thesis is a promising start for a vast research and development opportunities on the organic memristors and polymer conductors.

\end{doublespace}

