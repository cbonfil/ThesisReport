% Chapter 1
%!TEX root=../main.tex

\chapter{Introduction} % Main chapter title


\label{Chapter1} % For referencing the chapter elsewhere, use \ref{Chapter1} 

\lhead{Chapter 1. \emph{Introduction}} % This is for the header on each page - perhaps a shortened title

\section{Thesis Objectives}
\begin{doublespace}
The purpose of this thesis is to create a numerical simulation which captures the physics behind the operation of a coupled ionic-electronic device.. The proposed model can be used for researching the physics behind the actual device through further modeling or as a tool for designing coupled ionic electronic devices for a specific purpose. 

As an example of a coupled ionic-electronic device, a memristor model is created using only the essential components and physical effects. A numerical solver for \tjs{the coupled} drift diffusion and Poisson equations \tjs{is} developed and tested. A memristor simulation is created by modifying the drift diffusion solver to accommodate all the essential physical effects. Finally simulation results are compared with experimental results from a memristor fabricated at Carleton University.

\section{Thesis Overview}

After a brief introduction to coupled ionic electronic devices, the discovery of the memristor and challenges in memristor research is introduced in the \tjsr{background}{first} chapter. It emphasizes the need for better physical modeling for the memristor and the lack of simulation tools required for further analysis and research.

In chapter 2 , \tjs{the} Boltzmann Transport Equation\sout{s} (BTE) \tjsr{are}{is} introduced. This introduction is followed by the explanation of the charge transport mechanisms captured through \tjs{the} BTE. Additionally drift diffusion equations (\tjs{a} simplified form of BTE) which are used for the simulation of a coupled ionic-electronic device are discussed. Finally various analytic solutions to drift diffusion equations, such as charged particles moving over an infinite conductor and a PN junction, are derived and plotted. 

A numerical solution to the drift diffusion equations is formed in chapter 3 in order to calculate the movement of all the charged particles in a coupled ionic-electronic device. First a finite difference method, which is the basis for all the simulations in this thesis is introduced. Then it is applied to \tjs{the} drift diffusion equations. Additionally, a Poisson solver to be used with the drift diffusion equations is developed in order to calculate the electric field generated by charged particles and metal contacts. Various schemes for solving these differential equations as well as numerical and physical limitations are discussed in detail.

The structure of the organic memristor and the physical model used for the simulation is discussed in detail in chapter 4. The set of equations and boundary conditions necessary for the solution of both \tjs{the} drift diffusion equations and \tjs{the} Poisson solver are introduced. A special mechanism that was developed to limit the ion density at a particular area is discussed in detail. All the equations, additional physical effects and boundary conditions combined generate a set of equations to be solved for the simulation of the memristor.
 
In chapter 5, the numerical solver developed in chapter 3 is tested against analytic solutions as well as numerical solutions generated by a commercially available simulator \sout{called} \tjs{(}COMSOL Multiphysics\tjs{)}\cite{Comsol}. A mechanism to limit the maximum density of any particle, which is essential for memristor simulation, is developed and tested. 

After the stability and the accuracy of the simulation scheme is tested in chapter 6 a 1-D approximation for a 2-D memristor simulation is proposed and simulated. \tjsr{After}{Upon} demonstrating that the simulations produced reasonable results, the validity of the approximations made for 1-D and 2-D simulations are investigeted.  

A full 2-D simulation is discussed in chapter 7. \tjsr{A m}{M}emristor\tjs{s} with various PEDOT:PSS thicknesses are simulated and compared to each other. The simulation with thinnest PEDOT is compared to 1-D simulations made in the previous chapter. Finally memristor simulations are compared against experimental data.

The conclu\tjsr{sion}ding chapter summarizes the findings of the thesis, discusses the advantages and disadvantages of the proposed simulation methods and provides suggestions for improvements on the model and opportunities for further research.

\section{Contributions}

\begin{itemize}
  \item Numerical modelling of an organic memristor, which is a charge dependent resistor, using drift diffusion equations.  
  \item Development of a \tjsr{special mechanism}{numerical method} to \tjs{model physical processes} limit\tjs{ing} the amount of lithium ions that can accumulate inside the PEDOT:PSS.
  \item Simulation of two different ionic electronic structures, a single strip and a notched PEDOT:PSS, for various applied potentials.
  \item Model verification through experimental results.
\end{itemize}

\section{Background}

A coupled electronic-ionic device's behavior is determined by the movement \tjsr{i}{o}f mobile charged particles such as ions, holes and electrons inside a confined space. There are many examples of engineered and natural coupled electronic ionic devices. A good example of a natural ionic-electronic device is the ionic-channels in organic membranes which carry ions such as sodium, potassium, chlorine and calcium \cite{electrodiff}\cite{GramacidinChannel}. In electrical engineering coupled electronic-ionic devices are commonly proposed as an alternative to non-volatile memory devices due to ease of scalability and fabrication as well as low operation voltages \cite{IonicMemoryDev}\cite{NanoScaleMemEle}. They have also been proposed to be used as organic field effect transistors\cite{OrganicFet}. It is possible to find many physical models that are used simulate the behavior of an ionic electronic device using the drift-diffusion (aka. Poisson-Nernst-Planck) equations \cite{IonicMemoryDev}\cite{OrganicMemSim}\cite{NumericalPNP}.

As an example of a coupled ionic-electronic device, the modeling and the simulation of a memristor is studied in this thesis. The term memristor was first used by Leon Chua in 1971 in his paper called ``Memristor — The Missing Circuit Element'' \cite{chua}. He theorized that there is a fourth passive circuit element yet to be discovered in addition to resistor, capacitor and inductor. He said that we already know and use five out of six possible combinations that can be made out of four fundamental circuit variables, current \textit{I}, voltage \textit{V}, charge \textit{q} and flux linkage \textit{$\varphi$}. Chua claimed that there is a missing circuit element which produces a relationship between flux linkage (time integral of the potential) and charge. He introduced a new variable named memristance which has units of resistance and it is a function of charge. The relationship between the current and the potential of a memristor is calculated by replacing the resistance by memristance in ohms law \cite{memristance}:


\begin{equation}
v(t)=M(q(t))i(t)
\end{equation}

Theoretically a memristor retains its resistance in the absence of any power source. When a potential is applied, the resistance can be influenced by the direction and the magnitude of the current. If the current flows in one direction the resistance increases and if it flows in the other direction the resistance decreases. This produces an I-V response which looks like a pinched hysteresis curve. This response is the main characteristic of a memristor. 

This new element remained mostly a theory and it did not get a lot of academic attention until a group of researchers in Hewlett Packard developed a fully functioning memristor \cite{MisMem}. They successfully fabricated a nano scale memristor using $TiO_2$ (titanium dioxide). After the discovery in HP labs there was an increase of interest in different types of memristors due to their potential applications for data storage and the addition of learning capabilities into passive circuits \cite{AdaptiveMem} \cite{Synapse} \cite{CMOS}. 

Organic memristors are usually constructed on a larger scale than the \tjsr{ones}{$TiO_2$} fabricated \tjsr{in}{at the} HP labs. The memristor studied in this thesis is in the millimeter rather then nano meter scale and uses a conductor made out of a polymer called PEDOT:PSS. It is composed of two polymer chains, PEDOT and PSS attached together and conducts electricity via hole transport. Unfortunately the conduction mechanism is not perfectly understood and needs further research due to the complexity of the material \cite{PedotBook}.

Unlike a\tjs{n inorganic} semiconductor like silicon, the structure of a polymer is quite irregular \cite{PedotBook}. Polymers \tjsr{have}{are formed of} individual molecules with different chain lengths and a variable amount of defects. Additionally, they can be amorphous or partially crystalline and further differences occur through aging. The conduction and the electronic properties depend on the orientation of the polymer chains which \tjsr{can change in \textit{x, y} and \textit{z} directions}{be anisotropic and variable}. These irregularities in the structure makes these conducting polymers resemble amorphous inorganic semiconductors. Following the concept of charge transport in amorphous inorganic semiconductors, the conduction mechanism of conducting polymers is commonly explained by variable range hopping. This mechanism was first introduced by Mott\cite{Mott}. He proposed a model for charge transport in systems that are randomly disordered. In variable range hopping, charge transport occurs via jumps between available sites. Every charge carrier has a probability of jumping between two sites depending on its energy and the distance to the next available site. \tjsr{The temperature has a big impact on conductivity since}{The conductivity is strongly effected by temperature as} it changes the structure of the molecules and increases the energy of the charge carriers.

Apart from the temperature there are other methods to influence the conductivity of the PEDOT:PSS. It is possible to affect the conductivity by either doping or counter doping \cite{PedotDope}. Depending on the method used, doping can be reversible or permanent. When reversible doping is employed, the resulting device behaves like a memristor since its resistance becomes dependent on charge. All these effects make physical modeling, experimentation and simulation very challenging. 

\tjs{Memristor modeling is in its infancy}. \tjsr{There are a}{A} few compact models for inorganic memristors similar to the one produced in HP labs \tjs{have been proposed} \cite{ChuaSim}\cite{MemCircuitSim}\cite{MemTiO2}. There are \tjsr{a few}{also a number of modeling} studies o\tjsr{n}{f} organic memristors \cite{OrganicMemSim}\cite{OrganicMem}. \tjs{However, the work is slight leaving room} \tjsr{which are open}{} for further exploration. The physics behind the conduction mechanism and the changes in the conductivity due to doping requires further research and the development of a computer model can help in various ways. Theories that explain the movement of charged particles and their interactions with each other are difficult to formulate since it is hard to obtain required experimental data. A simulator can be a useful tool in testing various theories without having to set up and perform complicated experiments. Also once a solid understanding of the conduction mechanism is achieved and a reliable memsristor model is developed, simulations can be very useful to test different device structures and optimize them before fabrication. 


 \end{doublespace}