% Chapter 1

\chapter{Introduction} % Main chapter title

\label{Chapter1} % For referencing the chapter elsewhere, use \ref{Chapter1} 

\lhead{Chapter 1. \emph{Introduction}} % This is for the header on each page - perhaps a shortened title

\section{Thesis Objectives}

The purpose of this thesis is to create a numerical simulation which captures the phyiscs behind the operation of a memristor. The memristor simulation developed in this thesis could be used for researching the physics behind the actual device through further modeling or it could be used as a tool for designing memristors for a specific use. (maybe few more sentences)

\section{Thesis Overview}

Background section briefly talks about the history of memristor and how it was discovered before introducing the challenges and obstacles in research of memristors. It emphasizes the need for better physical modeling and lack of simulation tools required for further analysis and research.

In chapter 2 , Boltzmann Transport Equations (BTE) are shortly introduced. This introduction is followed by explanation of transport mechanisms captured through BTE. Additionally drift diffusion equations (simplified form of BTE) which will be used for the simulation of the memristor are discussed. Finally few analytic solutions to drift diffusion equations, such as charged particles moving over an infinite line and PN junction, are generated. 

A numerical solution to drift diffusion equations is formed to calculate the movement of all the charged particles in chapter 3. First finite difference method, which is the basis for all the simulations in this thesis,is introduced. Then it is applied to drift diffusion equations. Additionally, a Poisson solver to be used with drift diffusion equations is developed in order to calculate the electric field generated by charged particles and metal contacts. Also, boundary conditions necessary for the solution of both drift diffusion and Poisson solver are introduced. All the equations and boundary conditions combined generates a set of equations to be solved. Various schemes for solving these differential equations as well as the numerical and physical limitations are discussed in detail.
 
In chapter 4, the numerical solver developed in chapter 3 is tested against analytic solutions as well as a commercially available simulator called COMSOL Multiphysics. A mechanism to limit the maximum density of any particle, which is essential for memristor simulation, is developed and tested. 

After the stability and accuracy of the simulation scheme was tested in chapter 4, a memristor model is introduced in chapter 5. The difficulties in the simulation due to physical an numerical restrictions were discussed. A possible approximation 1-D approximation for a 2-D memristor simulation was proposed and simulated. After demonstrating that the simulations produced reasonable results, the validity of the approximations made for 1-D and 2-D simulations are investigeted.  

A full simulation 2-D is made in chapter 6. A memristor with various PEDOT thicknesses were simulated. The simulation with thinnest PEDOT is compared to 1-D simulations made in previous chapters. This chapter ends with comparison of simulation results with experimental data and shows how parameter fitting can be used to match expermental data.

The conclusion chapter summarizes the findings of the thesis, discusses advantages and disadvantages of the proposed simulation methods and provides suggestions for improvements on the model and opportunities further research.




\section{Background}
The term memristor was first used by Chua in 1971 in his paper called  "Memristor — The Missing Circuit Element" \cite{chua}. He theorized that there is a fourth passive circuit element yet to be discovered in addition to resistor,capacitor and inductor. He said that we already know and use five out of six possible combinations that can be made out of four fundamental circuit variables, current \textit{I}, voltage \textit{V}, charge \textit{q} and flux linkage \textit{$\varphi$}. Chua claimed that we were missing a circuit element which produces a relationship between flux linkage(time integral of the potential) and charge. He introduced a new variable named memristance which has units of resistance and it is a function of charge. The relationship between the current and the potential of a memristor is calculated by replacing the resistance by memristance in ohms law:

% \cite{chua} \cite{Dragica1} \cite{SG} \cite{NumModel} \cite{MisMem} \cite{SimChargeTrans} \cite{PedotBook} \cite{snowden} \cite{PedotCon}.

\begin{equation}
v(t)=M(q(t))i(t)
\end{equation}

Theoretically memristor retains its resistance in the absence of any power source. When a potential is applied the resistance can be influenced by the direction and the magnitude of the current. If the current flows in one dirfection resistance increases and if it flows in the other direction resistance decreases. This produces an I-V response which looks like a pinched hysteresis curve. This response is the main characteristic of a memristor. 

This new element remained mostly a theory and did not get much academic attention until a group of researchers in Hewlett Packard developed a fully functioning memristor. They successfully fabricated a nano scale memristor using $TiO_2$ (titanium dioxide). After the discovery in HP labs there was an increase of interest in different types of memristors due to their potential applications for data storage and addition of learning capabilities into passive circuits. 

Recently it has been suggested that organic based memristors could be fabricated (ref). This thesis will be concerned with such devices. They are usually constructed on a larger scale than the ones fabricated in HP labs. The memristor studied in this thesis is in millimeter rather then nano meter scale and uses a conductor made out of a polymer called PEDOT:PSS. It is made of two polymer chains, PEDOT and PSS attached together and conducts electricity via hole transport. Unfortunately the conduction mechanism is not perfectly understood and needs further research due to the complexity of the material(ref).

Unlike a semiconductors like silicon, the structure of a polymer is quite irregular. Polymers have individual molecules with different chain lengths and a variable amount of defects.
Additionally, they can be amorphous or partially crystalline and further differences occur through aging. The conduction and electronic properties depend the orientation of polymer chains which can change in x,y and z directions. These irregularities in the structure makes these conducting polymers resemble amorphous inorganic semiconductors. Following the concept of charge transport in amorphous inorganic semiconductors, the conduction mechanism of conducting polymers is commonly explained by variable range hopping. This mechanism was first introduced by Mott in 1968. He proposed a model for charge transport in systems that are randomly disordered. In variable range hopping charge transport occurs via jumps between available sites. Every charge carrier has a probability of jumping between two sites depending on its energy and the distance to the next available site. Temperature has a big impact on conductivity since it changes the structure of the molecules and increases the energy of charge carriers.

Apart from temperature there are other ways to change the conductivity of PEDOT. It is possible to affect conductivity by either doping or counter doping. Depending on the method used, doping can be reversible or permanent. When reversible doping is employed, resulting device behaves like a memristor since its resistance is now dependent on charge. All these effects make physical modeling, experimentation and simulation very challenging. 

There are few approximate analytic solutions for inorganic memristors similar to the one produced in HP labs. These solutions are mostly based on parameter fitting and some basic physics. Studies on organic memristors are even more sparse than studies on inorganic ones. The physics behind the conduction mechanism and the changes in conductivity due to doping requires further research and development of a computer model can help in various ways. First of all it is very difficult to get experimental data on the movement of ions and holes inside PEDOT due its disordered structure. The thickness of PEDOT:PSS (usually in $\mu$m scale) complicates the tracing of the ion in all dimensions. Theories that explain the movement of charged particles and their interactions with each other are difficult to formulate since it is hard to obtain required experimental data. A simulation can be a useful tool in testing various theories without having to set up and perform complicated experiments. Also once a solid understanding of the conduction mechanism has been achieved, simulations can be very useful to test different device structures and optimize them before fabrication. 

(add lack of simulations for memristor)
 