

\chapter{Numerical Solution of Drift-Diffusion Equation} % Main chapter title

\label{Chapter3} % Change X to a consecutive number; for referencing this chapter elsewhere, use \ref{ChapterX}

\lhead{Chapter 3. \emph{Numerical Solution for Drift-Diffusion Equations}} % Change X to a consecutive number; this is for the header on each page - perhaps a shortened title
\begin{doublespace}


As discussed in the introduction, memristor simulations with detailed physical effects are almost non existent. Memristor simulation developed in this thesis requires certain properties ,such as limitation of particle density or combination of 1-D and 2-D simulations, which may not be available in most commercial simulators. It is also possible that other unusual properties that are not studied in this thesis will be needed for further research. In this chapter a method for solving drift diffusion equations as well as Poisson's equation are developed based on finite difference method. This fully functioning simulator will have the needed flexibility for current and future research opportunities.  


\section{Finite Difference Method}
There are many different methods that can be used to solve drift diffusion equations such as finite elements, finite difference or meshless methods. Finite difference was chosen as an appropriate method for this thesis due to its simplicity which allows straightforward implementation of unusual physical properties. Finite difference uses an approximation for the derivative of a function based on the mathematical definition of the derivative\cite{numerical2}.

\begin{equation}
\frac{df}{dx}=\lim\limits_{h \rightarrow 0} \frac{f(x+h)-f(x)}{h}
\end{equation}

It is possible to obtain a numerical approximation for the first derivative of a function by dropping the limit and assuming that h is small enough that the numerical derivative is reasonably close to the actual derivative. As h gets smaller the approximation becomes more and more accurate. The difference between the calculated value and the real value is called the truncation error and it is captured using $O(h^n)$ notation. n signifies the order of h which determines how fast the approximation is approaching the real solution as h decreases.  
\begin{equation}
\frac{df}{dx}=\frac{f(x+h)-f(x)}{h} + O(h)
\label{numdif}
\end{equation}

It is possible to uniformly discretize the entire region over which a function is defined in order to calculate its derivative. The first step is the division of the region over which the function is defined into \textit{n-1} segments. This creates \textit{n} number of points. Then the length of each segment is defined using the following relationship:

\begin{equation}
h=\frac{L}{n}
\end{equation}

A function is defined at the edge of every segment. All the points can be labeled consecutively, $x_0,x_1,x_2$ ... $x_{n-1}$ where $x_i=ih$. The function is discretely defined on \textit{$f_{i}=f(x_{i})$} where\textit{ i=0,1,2..n-1}. It is possible to use equation  \ref{numdif} to discretely calculate the first derivative of the function with respect to x.

\begin{equation}
\frac{df(x_i)}{dx}=\frac{f(x_{i+1})-f(x_i)}{h} + O(h)
\end{equation}
Above equation is called forward difference because the derivative for point $x_i$ was calculated using the point that is coming right after it, $x_{i+1}$.
\begin{equation}
f^{'}_i=\frac{f_{i+1}-f_i}{h}+ O(h)
\end{equation}

Forward difference is not the only way to calculate a numerical derivative. Here are a few other ways calculate the same derivative by using of different points.

\begin{equation}
f^{'}_i=\frac{f_{i}-f_{i-1}}{h}+ O(h)
\label{bdif}
\end{equation}

\begin{equation}
f^{'}_{i+\frac{1}{2}}=\frac{f_{i+1}-f_i}{h}+ O(h^2)
\label{cdif}
\end{equation}

Equation \ref{bdif} is called backward difference and equation \ref{cdif} is called central difference. One important aspect to note here is that in the central difference formula the derivative falls exactly in the middle of two points. It also gives more accurate results using the same number of points as forward and backward difference. 

Using finite difference formulas it is possible to construct higher order derivatives. A formula for a second order derivative at point $x_i$ using central difference can be calculated using a first order derivative on $x_{i-\frac{1}{2}}$, $x_i$ and $x_{i+\frac{1}{2}}$  

\begin{equation}
f_{i+\frac{1}{2}}^{'}=\frac{f_{i+1}-f_{i}}{h}
\label{forwardd}
\end{equation}

\begin{equation}
f_{i-\frac{1}{2}}^{'}=\frac{f_{i}-f_{i-1}}{h}
\label{backwardd}
\end{equation}

\begin{equation}
f^{'}_{i}=\frac{f_{i+\frac{1}{2}}-f_{i-\frac{1}{2}}}{h}
\label{2ndord}
\end{equation}

The second order derivative is constructed by taking the second derivative of the last function. Then equations \ref{forwardd} and \ref{backwardd} are placed into \ref{2ndord}.

\begin{equation}\nonumber
f^{''}_{i}=\frac{f_{i+\frac{1}{2}}^{'}-f_{i-\frac{1}{2}}^{'}}{h}
\end{equation}

\begin{equation}\nonumber
f^{''}_{i}=\frac{\frac{f_{i+1}-f_{i}}{h}-\frac{f_{i}-f_{i-1}}{h}}{h}
\end{equation}

\begin{equation}\nonumber
f^{''}_{i}=\frac{f_{i+1}-f_{i}-f_{i}+f_{i-1}}{h^2}
\end{equation}

Second order derivative takes the following form:

\begin{equation}
f^{''}_{i}=\frac{f_{i+1}-2f_{i}+f_{i-1}}{h^2}+O(h^2)
\label{fdc2}
\end{equation}

Overall these finite difference equations are enough to solve drift diffusion equations. Even though all the derivations are done in 1-D it is trivial to extend them to other dimensions. This method can be used to solve Poisson's equation and drift diffusion equations.

\clearpage

\section{Poisson Solver}

Poisson's equation needs to be solved before drift diffusion equations in order find the potential distribution as well as the electric field inside the device. In order to solve for electric field and potential Poisson's equation is simplified through assumptions and then finite difference is used to solve this simplified equation\cite{smith1985numerical}. The first step of simplification is assuming that the permittivity is isotropic.
 
\begin{equation}
\nabla \cdot  (\varepsilon \nabla V)=-\rho
\end{equation}

\begin{equation}
\nabla \cdot  (\varepsilon \nabla V)=\varepsilon  \nabla^2 V
\end{equation}

Dividing both sides by permittivity and expanding the left hand side,,

\begin{equation}
 \nabla^2 V =-\frac{\rho}{\varepsilon}
 \label{Poissons}
\end{equation}

\begin{equation}
 \nabla^2 V =\frac{\partial^2 V}{\partial^2 x}+\frac{\partial^2 V}{\partial^2 y}
\end{equation}

After discretizing the electric potential over a 2-D uniform grid and using the second order central finite difference formula \eqref{fdc2} laplacian of the electric potential can be calculated using:

\begin{equation}
 \nabla^2 V_{i,j}=\frac{V_{i+1,j}-2V_{i,j}+V_{i-1,j}}{\Delta x^2}+\frac{V_{i,j+1}-2V_{i,j}+V_{i,j-1}}{\Delta y^2}
 \label{nablafd}
\end{equation}

Since the grid is uniform, the distance between two nodes in x and y directions are equal therefore only one variable is needed to represent the distance between two points.

\begin{equation}
\Delta=\Delta x =\Delta y
\label{delta}
\end{equation}

Net charge density and the permittivity is also discretized over the same uniform mesh. Discretized form of Poisson's equation is generated by combining equations \ref{Poissons} ,\ref{nablafd} and \ref{delta}.

\begin{equation}
 \nabla^2 V_{i,j}=\frac{V_{i-1,j}+V_{i,j-1}-4V_{i,j}+V_{i+1,j}+V_{i,j+1}}{\Delta^2}=-\frac{\rho_{i,j}}{\varepsilon_{i,j}}
\end{equation}

This equation can be rearranged into the form below:
\begin{equation}
\varepsilon_{i,j}(V_{i-1,j}+V_{i,j-1}-4V_{i,j}+V_{i+1,j}+V_{i,j+1})=-\Delta^2\rho_{i,j}
\label{discrete_poisson}
\end{equation}

Equation \ref{discrete_poisson} is valid for almost all the nodes in the system except two cases, boundary nodes and interface nodes. There are two different types of boundary conditions. The first one is Dirichlet boundary condition which forces a particular value for the potential at the boundary.

\begin{equation}
V_{i,j}=V_{b}
\label{dirichlet}
\end{equation}

Where $V_{b}$ is the value of the potential at the boundary. The other possible boundary condition is called Neumann boundary condition which states that the derivative of the potential at the boundary is zero. This gives the following equation:

\begin{equation}
\frac{\partial V}{\partial x}=\frac{V_{i+1,j}-V_{i,j}}{\Delta}=0
\label{neumannx}
\end{equation}

So for a boundary in y direction:

\begin{equation}
V_{i+1,j}=V_{i,j}
\label{neumanny}
\end{equation}

Neumann boundary condition in x direction is obtained using the same procedure.

\begin{equation}
V_{i,j+1}=V_{i,j}
\end{equation}

Combining the equations above (\ref{discrete_poisson}, \ref{dirichlet},\ref{neumannx} and \ref{neumanny}) it is possible to turn Poisson's equation, which is a second order differential equation, into a linear set of coupled algebraic equation.

\begin{equation}
D_{2}\vec{V}=-\Delta^2\vec{\rho}-\vec{V_b}
\end{equation}

$D_{2}$ is the Laplace operator converted into a matrix using the finite difference method. One can easily get the potential distribution by simply solving this matrix equation. Due to the nature of the problem the resulting matrix is quite sparse and using a sparse LU rather than a regular LU decomposition increases the computational efficiency. Additionally, LU decomposition only needs to be performed once. Since the equation is static L and U matrices can be reused for all the solutions following the initial one. 

\begin{equation}
\vec{V}=D_{2}^{-1}(-\Delta^2\vec{\rho_{i,j}}-\vec{V_b})
\end{equation}

After solving for the potential distribution it is straightforward to calculate the electric field distribution discretely using the relationship between electric field and electric potential \eqref{Efield} and central difference equation \eqref{cdif}. 

\begin{equation}
\vec{E^x_{i,j}}=-\frac{V_{i+1,j}-V_{i-1,j}}{2\Delta}
\end{equation}

\begin{equation}
\vec{E^x_{i,j}}=-\frac{V_{i,j+1}-V_{i,j-1}}{2\Delta}
\end{equation}

\clearpage
\section{Current Density Equations}
 Both drift and diffusion currents can be calculated over the entire grid. Drift current does not involve any differentials but it is a function of electric field and the diffusion current can be calculated using first order central difference\cite{Dragica1}. The current density is calculated in such a way that it falls between two points which simplifies the application of the boundary conditions.
\begin{equation}
J^x_{i+\frac{1}{2},j,k}=q\mu_n n_{i+\frac{1}{2},j,k} E^x_{i+\frac{1}{2},j,k}+D_n \frac{n_{i+1,j,k}-n_{i,j,k}}{\Delta}
\end{equation}
The electric field was calculated exactly on the nodes and linear interpolation was used in order to get a value between the nodes. The same argument is also valid for particle densities \textit{p} and \textit{n}. They were defined on the nodes but they were linearly interpolated to be used in current density equations.

\begin{equation}\nonumber
n_{i+\frac{1}{2},j,k}=\frac{n_{i+1,j,k}+n_{i,j,k}}{2}
\end{equation}
\begin{equation}\nonumber
E^{x}_{i+\frac{1}{2},j,k}=\frac{E^y_{i+1,j,k}+E^y_{i,j,k}}{2}
\end{equation}

Current density in y direction is calculated following the same method:

\begin{equation}
J^y_{i,j+\frac{1}{2},k}=q\mu_n n_{i,j+\frac{1}{2},k} E^y_{i,j+\frac{1}{2},k}+D_n \frac{n_{i,j+1,k}-n_{i,j,k}}{\Delta}
\end{equation}
\begin{equation}\nonumber
n_{i,j+\frac{1}{2},k}=\frac{n_{i,j+1,k}+n_{i,j,k}}{2}
\end{equation}
\begin{equation}\nonumber
E^{y}_{i,j+\frac{1}{2},k}=\frac{E^y_{i,j+1,k}+E^y_{i,j,k}}{2}
\end{equation}


\clearpage
\subsection{Continuity Equation}
The continuity equation is needed to calculate a transient solution for the drift diffusion equations. The equation is simple to discretize using the finite difference method. There are two terms that need to be discretized, a first order derivative in time and a first order derivative in space. First the divergence term in the equation \eqref{conn} needs to be evaluated.

\begin{equation}
\nabla \cdot J=\frac{\partial J}{\partial x}+\frac{\partial J}{\partial y}=\frac{d J_x}{d x}+\frac{d J_y}{d y}
\end{equation}

It is possible to replace the derivative with central finite difference terms.
\begin{equation}
\frac{d J_x}{d x}=\frac{J^x_{i+\frac{1}{2},j,k}-J^x_{i-\frac{1}{2},j,k}}{h}
\end{equation}
\begin{equation}
\frac{d J_y}{d y}=\frac{J^y_{i,j+\frac{1}{2},k}-J^y_{i,j-\frac{1}{2},k}}{h}
\end{equation}
\begin{equation}
\nabla \cdot J_{i,j,k}=\frac{J^x_{i+\frac{1}{2},j,k}-J^x_{i-\frac{1}{2},j,k}}{h}+\frac{J^y_{i,j+\frac{1}{2},k}-J^y_{i,j-\frac{1}{2},k}}{h}
\label{delJ}
\end{equation}

This is the general form of the divergence of the current density. These set of equations can be placed in to a matrix which is a linear function of particle density at time t.

\begin{equation}
B = \nabla \cdot J_k 
\label{fd_div}
\end{equation}

The time derivative can be replaced by a forward or backward finite difference terms respectively.

\begin{equation}
\frac{\partial  \vec{n}_k}{\partial t}=\frac{ \vec{n}_{k+1}-\vec{n}_k}{\Delta t}
\label{forwardtime}
\end{equation}

\begin{equation}
\frac{\partial \vec{n}_k}{\partial t}=\frac{ \vec{n}_k- \vec{n}_{k-1}}{\Delta t}
\label{backwardtime}
\end{equation}

It is possible to find a numerical transient solution for the drift-diffusion problem by combining finite difference form of the time derivative (\eqref{forwardtime} or \eqref{backwardtime}) and the divergence of the current density equations \eqref{fd_div}.

Forward difference approximation can be used to get an explicit solution for continuity equation.

\begin{equation}\nonumber
\frac{ \vec{n}_{k+1}-\vec{n_k}}{\Delta t}=B
\end{equation}

\begin{equation}
\vec{n}_{k+1}=\vec{n_{k}}+\Delta t B
\label{explicit}
\end{equation}

Both forward and backward difference formulas work sequentially in order to generate a transient solution.The solution from the previous time step is needed to calculate the solution for the next time step. The forward difference gives an explicit solution which has a few advantages. This solution can be implemented, without forming any matrices by directly calculating the divergence of the current density for each node and then marching through time using equation \ref{explicit}. Additionally, unlike backward difference, there are no equations to be solved for every time step. These two properties ease the computational load of the problem and speed up the solution process. Unfortunately this scheme has very strict stability conditions which have to be met in order to produce a solution.

Backward difference can be used instead of forward difference to get an implicit solution.
\begin{equation}\nonumber
\frac{ \vec{n}_{k}-\vec{n}_{k-1}}{\Delta t}=\frac{1}{q}(qB)
\end{equation}
\begin{equation}\nonumber
\vec{n}_{k}-\Delta t B =\vec{n}_{k-1}
\end{equation}

Since all the equations in B matrix are a linear functions on n, it can be separated into two terms, $B=Cn$.
\begin{equation}\nonumber
\vec{n}_{k}-\Delta t C\vec{n}_{k-1} =\vec{n}_{k-1}
\end{equation}
\begin{equation}
\vec{n}_k=(I-\Delta t C)^{-1}\vec{n}_{k-1}
\end{equation}

This solution needs a matrix inversion every time step but it is unconditionally stable if it is not used together with Poisson's equation. The decision to use an implicit or an explicit solution is not very simple and it will be discussed in detail in the section following boundary conditions.
\clearpage


\subsection{Boundary Conditions}

For the drift diffusion problem solved in this thesis there are two different possibilities for boundary conditions on moving particles. No flow conditions are imposed on the current density equations where metal contact condition is imposed on the particle densities. There are two sub types of no flow boundary conditions used during simulation, regular and dependent no flow boundary conditions. A regular no flow condition is used when the particles cannot go past a certain boundary. This is achieved by setting the particle flow at any boundary to zero. 

Dependent no flow boundary condition is a term that is used to describe a boundary condition which can be a function of any variable such as temperature, particle or charge density. For the lithium ions this condition is made a function of lithium density and it is used not only for the boundaries but also for all the points inside the PEDOT:PSS. Basically, during simulation, at any point inside PEDOT:PSS if the lithium density goes over a certain limit then that point turns into a no flow wall for the lithium ions as long as there is a net influx of particles. The boundary condition is removed if the lithium density at that point will go below the set limit at the next time step during transient.  

To simulate the metal contacts it is assumed that they hold an infinite amount of positive and negative charges and the boundary is always charge neutral. For example for holes, electrons, positive and negative doping it is assumed that at the boundary positive charge concentration will be equal to the negative charge concentration. 

\begin{equation}
N_{D} + p=N_{A} + n
\label{chargeneutrality}
\end{equation}

For a semiconductor holes and electrons have to obey the mass action law.

\begin{equation}
np=n_i^2
\label{massaction}
\end{equation}

$n_i$ is the concentration of the semiconductor at equilibrium before getting doped. Solving \eqref{massaction} and \eqref{chargeneutrality} together results in the equation below:

\begin{equation}
p=\frac{1}{2}(N_A - N_D + \sqrt{(N_A - N_D)^2+4n_i^2})
\label{nbound}
\end{equation}

Once the hole concentration is obtained it is possible to calculate the electron concentration using the mass action law.

\begin{equation}
n=\frac{n_i^2}{p}
\label{pbound}
\end{equation}

Figures \ref{bc_hole}, \ref{bc_lithium} and \ref{bc_perchlorate} show the boundary conditions used for holes, lithium and perchlorate ions.

\begin{figure}[!htp]
\centering
\includegraphics[scale=0.6]{bc_hole}
\caption{Boundary conditions for the holes } 
\label{bc_hole}
\end{figure}

\begin{figure}[!htp]
\centering
\includegraphics[scale=0.6]{bc_lithium}
\caption{Boundary conditions for the lithium ions } 
\label{bc_lithium}
\end{figure}

\begin{figure}[!htp]
\centering
\includegraphics[scale=0.6]{bc_perchlorate}
\caption{Boundary conditions for the perchlorate ions } 
\label{bc_perchlorate}
\end{figure}

During simulation, the application of this boundary condition differs from the no flow boundary which is applied implicitly. All the boundaries are simulated using a no flow condition but their values are set to the appropriate values at the end of every time step. The lack of charge is compensated and excess charge is taken off by the metal contact. The difference between the boundary value of the charge density and its actual value is used to calculate the derivative of the current density with respect to time. The current in and out of the device is calculated by integrating the derivative of the current density over time. 

\begin{equation}
\frac{dn}{dt}=\frac{n_{cr}-n_{ct}}{\Delta t}
\end{equation}

$n_{cr}$ is the excess carrier density and $n_{ct}$ is the equilibrium carrier density at the contact. Using these boundary conditions for holes and electrons and the following two equations it is possible to calculate incoming and outgoing currents. 

\begin{equation}
Q=An
\label{charge_charge_density}
\end{equation} 

\begin{equation}
I=\frac{dQ}{dt}
\label{current_charge}
\end{equation} 
 

Equation \ref{charge_charge_density} is an approximate relationship between charge and charge density where Q is total charge, A is the area holding that charge. Equation \ref{current_charge} is simply the general definition of current. By combining these two equations it is possible to derive a formula for calculating the current leaving or entering the device at any metal contact.

\begin{equation}
I=A \frac{dn}{dt}
\label{current_charge_density}
\end{equation}


\section{Stability and Computational Efficiency}
Before discussing  the numerical limitations of solving drift diffusion equation via finite difference it is important to look into physical limitations of the problem. These limitations persist no matter what kind of numerical scheme is employed to solve drift diffusion equations.

\subsection{Physical Limitations}
Debye length is the length over which mobile charge carriers screen out an external electric field and it determines how steeply charges will accumulate over a certain distance.
\begin{equation}
L_D=\sqrt{\frac{\varepsilon V_{th}}{q n}}
\label{debye}
\end{equation}
Debye length limits how coarse the grid can be since the distribution of the charge density needs to be accurately captured. As it can be seen from the formula above the higher the charge density is the steeper the charge will accumulate. This behavior is also appears in the analytic solution provided in section 2.2.1 in chapter 2. The accumulation of the charged particle at the wall becomes steeper as the electric field strength increases. The debye length can become a major problem for device sizes in millimeter range and high charge densities since the mesh density needs to be extremely high.

The amount if time it takes for charge fluctuations to disappear is called Dielectric relaxation time. It limits the maximum time step of a simulation since the fluctuations that are not properly resolved over time will make the simulation unstable.

\begin{equation}
t_{dr}=\frac{\varepsilon}{q n \mu}
\label{tdr}
\end{equation}

Dielectric relaxation time is only important when electric potential is highly affected by redistribution of charge over time. Otherwise it has minimal impact on the stability of the problem.
\subsection{Numerical Limitations}

There are also numerical limits which can affect convergence and stability of a solution when using an explicit scheme. These are called Courant-Friedrichs-Lewy (CFL)conditions \cite{NumModel}. CFL conditions for pure diffusion and pure drift are shown in the following equations.

\begin{equation}
\frac{\Delta ^2}{2 D_n}>\Delta t
\label{CFL_Diff}
\end{equation}

Above condition is for pure diffusion and it restricts the maximum time step. Following condition is for drift dominated systems:

\begin{equation}
\frac{2 \Delta }{\mu E}>\Delta t
\label{CFL_Drift}
\end{equation}

This is the second numerical restriction on the simulation. The condition for drift depends on the electric field therefore it needs to be satisfied as the electric field changes over time during simulation.

Both physical and numerical constraints have to be evaluated and mesh density and time step needs to be selected in order to satisfy all these conditions discussed above. Particularly mesh density has a very strong effect on the accuracy, stability and computational efficiency of the simulation. Increasing the mesh density increases the computational time needed to calculate every time step since there are more points. Additionally because of the CFL condition for diffusion, time step is related to the square of the mesh size. This means that maximum allowed step size decreases much quicker than the mesh density. Also, increasing charge density can decrease the maximum mesh size to a very small value. This can be fixed by using a non uniform mesh which can dramatically decrease the amount points needed for the simulation. This is usually not very straightforward to implement in a finite difference scheme. Also a small mesh size requires small time steps. This cannot be avoided through non uniform meshing. Both numerical and physical constraints for the memristor simulation are further discussed in chapter 5.

\subsection{Explicit vs. Implicit Solution}

Overall explicit and implicit solutions have their advantages and disadvantages. Choosing one over the other requires careful analysis of the problem. Implicit solution by itself is unconditionally stable therefore it can support very large time steps without any stability issues. However with increased time step, the accuracy of the transient solution decreases but the steady state solution does not get affected. So for steady state solutions it is better to use an implicit method which can reach steady state very quickly. This advantage disappears when particle densities are high enough to affect the electric field and Poisson's equation needs to be solved for every time step. In this scenario the maximum step size is determined by dielectric relaxation time which can be around the same order as CFL conditions or even smaller. Since the time step is going to be around the same order for both implicit and explicit methods it is reasonable to use the explicit one because it is computationally less expensive.

Usually implicit solution is preferable when there is no coupling between Poisson's equation and the drift diffusion equations and the transient response is not very important. Explicit solution has an edge over the implicit solution due to its lower computational resource requirement when the equations are coupled and the time steps for both schemes are restricted to fairly small values. For memirstor simulation, drift diffusion equations are strongly coupled with Poisson's equation. For this reason all memristor simulations in this thesis use explicit time stepping. 


\clearpage
\section{Simulation Procedure}
Different equations and schemes that are used to solve drift diffusion and Poisson's equation is shown over the past few sections. Using all this information it is possible create a general approach to solve a drift diffusion problem. Following equations are used to simulate the ion and hole movement and the changes in electric field in an organic memristor using finite difference method:

\begin{equation}
\nabla \cdot  (\varepsilon \nabla V)=-q(p-n+N_{D}^{+}-N_{A}^{-})
\end{equation}
\begin{equation}
\vec{J_p}=q\mu_p p \vec{E}-q D_p \nabla p
\end{equation}
\begin{equation}
\vec{J_{N_{A}^{-}}}=q\mu_{N_{A}^{-}} N_{A}^{-} \vec{E}+q D_{N_{A}^{-}} \nabla N_{A}^{-}
\end{equation}
\begin{equation}
\vec{J_{N_{D}^{+}}}=q\mu_{N_{D}^{+}} N_{D}^{+} \vec{E}-q D_{N_{D}^{+}} \nabla N_{D}^{+}
\end{equation}
\begin{equation}
\frac{\partial p}{\partial t}=-\frac{1}{q}\nabla \cdot \vec{J_p}
\end{equation}
\begin{equation}
\frac{\partial N_{A}^{-}}{\partial t}=\frac{1}{q}\nabla \cdot \vec{J_{N_{A}^{-}}}
\end{equation}
\begin{equation}
\frac{\partial N_{D}^{+}}{\partial t}=-\frac{1}{q}\nabla \cdot \vec{J_{N_{D}^{+}}}
\end{equation}

The geometry and physical properties of the problem as well as all initial and boundary conditions need to be defined at the beginning of the solution process.The initialization sets up the first time step of the problem at $t=0$. Once this first step is done it is possible generate the required vectors and matrices and solve the problem for the next time steps, $t=t_i$. 

The solution process starts by solving Poisson's equation using the charge distribution at current time step. Once it is solved, the electric field distribution is calculated and used in drift diffusion equations to calculate current density distribution. Next, explicit time stepping is used to determine carrier density in the next time step. Finally once the carrier distribution for the next step is calculated it is possible check for a stopping criterion. If this criterion is not met then the whole process will start all over again. If the charge concentration is so small that the equations are decoupled then it is possible to skip solving Poisson's equation for the rest of the simulation which speeds up the solution process.

 There are two different criteria that can be used to decide whether to finalize the simulation process or not. The simulation can stop if it reaches a certain point in time. This is quite simple since the current time can be checked and if it is equal or greater than the required simulation time then the simulation can be stopped. It is also possible to simulate until the simulation reaches steady state. This can be determined by comparing the current carrier distributions with distribution at the previous time step. If the difference is very small then the time derivatives of the carrier densities are very close to zero and the simulation has reached steady state therefore the simulation process can be stopped. The flowchart in figure \ref{flowchart} summarizes the solution procedure.


\clearpage

\begin{figure}
\centering
\includegraphics[scale=1.3]{flowchart}
\caption{Finite Difference Drift-Diffusion Scheme Flowchart} 
\label{flowchart}
\end{figure}

\end{doublespace}
