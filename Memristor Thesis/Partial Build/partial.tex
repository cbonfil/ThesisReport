%%%%%%%%%%%%%%%%%%%%%%%%%%%%%%%%%%%%%%%%%
% Thesis 
% LaTeX Template
% Version 1.3 (21/12/12)
%
% This template has been downloaded from:
% http://www.latextemplates.com
%
% Original authors:
% Steven Gunn 
% http://users.ecs.soton.ac.uk/srg/softwaretools/document/templates/
% and
% Sunil Patel
% http://www.sunilpatel.co.uk/thesis-template/
%
% License:
% CC BY-NC-SA 3.0 (http://creativecommons.org/licenses/by-nc-sa/3.0/)
%
% Note:
% Make sure to edit document variables in the Thesis.cls file
%
%%%%%%%%%%%%%%%%%%%%%%%%%%%%%%%%%%%%%%%%%

%----------------------------------------------------------------------------------------
%	PACKAGES AND OTHER DOCUMENT CONFIGURATIONS
%----------------------------------------------------------------------------------------

\documentclass[11pt, a4paper, oneside]{Thesis} % Paper size, default font size and one-sided paper


\graphicspath{{./Pictures/}} % Specifies the directory where pictures are stored
\usepackage{pgfplots}
\usepackage{morefloats}
\usepackage{enumerate}
\usepackage{setspace}
\usepackage{tikz}
\usetikzlibrary{shapes,shadows,arrows}
\usepackage{rotating}
\usepackage{pdflscape}
\usepackage{ulem}
\newcommand\redout{\bgroup\markoverwith
{\textcolor{red}{\rule[.5ex]{2pt}{1pt}}}\ULon}
% \usepackage[usenames]{color}
\usepackage{color}

\usepackage[square, numbers, comma, sort&compress]{natbib} % Use the natbib reference package - read up on this to edit the reference style; if you want text (e.g. Smith et al., 2012) for the in-text references (instead of numbers), remove 'numbers' 
\hypersetup{urlcolor=blue, colorlinks=true} % Colors hyperlinks in blue - change to black if annoying
\title{\ttitle} % Defines the thesis title - don't touch this

\newcommand{\tjs}[1] {\textcolor{black}{#1}}
\newcommand{\tjsr}[2] {\redout{}\textcolor{black}{#2}}

\begin{document}

\frontmatter % Use roman page numbering style (i, ii, iii, iv...) for the pre-content pages

\setstretch{1.3} % Line spacing of 1.3

% Define the page headers using the FancyHdr package and set up for one-sided printing
\fancyhead{} % Clears all page headers and footers
\rhead{\thepage} % Sets the right side header to show the page number
\lhead{} % Clears the left side page header

\pagestyle{fancy} % Finally, use the "fancy" page style to implement the FancyHdr headers

\newcommand{\HRule}{\rule{\linewidth}{0.5mm}} % New command to make the lines in the title page

% PDF meta-data

%----------------------------------------------------------------------------------------
%	TITLE PAGE
%----------------------------------------------------------------------------------------

\begin{titlepage}
\begin{center}
\LARGE
\textbf{The Drift Diffusion Simulation of Coupled Ionic-Electronic Devices}
\vspace{6 mm}
\large 
\begin{center}
by 
\vspace{6 mm}
\end{center}
\large  
Cem Bonfil, B. Eng.
\end{center}



\large  
\begin{center}
A thesis submitted to the
Faculty of Graduate and Postdoctoral Affairs
in partial fulfillment of
the requirements for the degree of
\end{center}

\vspace{6 mm}
\large  
\begin{center}
Master of Applied Science
\end{center}
\large  
\begin{center}
in
\end{center}
\large  
\begin{center}
Electrical and Computer Engineering
\end{center}

\vspace{6 mm}

\large  
\begin{center}
Ottawa-Carleton Institute for Electrical and Computer Engineering
\end{center}

\large  
\begin{center}
Department of Electronics
\end{center}
\large  
\begin{center}
Carleton University
\end{center}
\large  
\begin{center}
Ottawa, Ontario, Canada
\end{center}

\vspace{6 mm}
\large  
\begin{center}
\copyright 2014, Cem Bonfil
\end{center}

\end{titlepage}

\newpage % Start a new page

\Large
\textbf{Abstract}
\newline
\begin{doublespace}
The purpose of \tjs{the work presented in this} thesis was to create \tjs{a} numerical simulation scheme for a coupled ionic electronic device and use it to simulate a memristor which is based on an organic conductor called poly(3,4 ethylenedioxythiophene):polystyrenesulfonate (PEDOT:PSS). The memristor that was modeled consists of \tjsr{three major parts,}{} a thin PEDOT:PSS strip with a metal contact on both sides and a drop of an electrolyte solution with lithium and perchlorate ions. The conductivity of the memristor changes when lithium ions in the electrolyte dedopes the PEDOT:PSS by bonding with PSS polymers. A numerical drift diffusion and a Poisson solver was implemented with special features to model the physical properties of the memristor. \tjs{The d}eveloped simulation algorithm was tested using \tjsr{the}{} analytical solutions to the drift diffusion equations and Poisson's equation. 1-D and 2-D memristors were simulated using various applied potentials and the results were compared to each other. Both 1-D and 2-D simulations were able to capture the essential physical effects. The comparison of 2-D simulations and experimental results showed that proposed model worked as expected and produced results that were similar to an actual memristor. The work presented in this thesis showed promising results for the simulation of a memristor which can be improved in the future by additional modeling of the charge transport mechanisms.
\end{doublespace}
\normalsize
\begin{doublespace}

\end{doublespace}